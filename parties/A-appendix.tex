\section{Using Gitlab CI}

Here is an example of \code{.gitlab-ci.yml} for building the PDF for online and print:

\begin{minted}{yaml}
    compile_pdf:
    image: aergus/latex
    script:
      - p0=$(pwd) 
      - apt install -y python-setuptools 
      - cd / && git clone git://github.com/hugomaiavieira/pygments-style-github.git
      - cd pygments-style-github && python setup.py install 
      - cd $p0
      - make && make bcompile && make glossaries && make && make
      - mv report.pdf report-online.pdf
      - sed 's/oneside/twoside/' report.tex 1<> report.tex
      - make && make
      - mv report.pdf report-print.pdf
    artifacts:
      paths:
        - report-online.pdf
        - report-print.pdf
    only:
      - master     
\end{minted}


\section{Glossary}

Using a glossary is easy in latex:
\begin{itemize}
    \item use the package \code{glossaries};
    \item create a \code{glossaries.tex} file (see Listing~\ref{lst:glossaries});
    \item include the file in the report and print it using \code{\\printglossary};
    \item to compile the glossary, use the terminal command \code{makeglossaries -d out}.
\end{itemize}


Here are some command you can add to the \code{header.tex}. This will automatically color the glossary entries in blue in the text (provided you use those commands):

\begin{minted}{latex}
    %% glossary / vocabulary

    % term in blue (but no glossary links)
    \newcommand{\ddef}[1]{\textcolor{blue}{#1}}
    % glossary links in blue
    \newcommand{\ddefg}[1]{\textcolor{blue}{\hypersetup{linkcolor=blue}\gls{#1}}}
    \newcommand{\ddefgpl}[1]{\textcolor{blue}{\hypersetup{linkcolor=blue}\glspl{#1}}}
\end{minted}

I also customized a bit the glossary style using this code (you can put it at the top of the \code{glossary.tex} file for example):

\begin{minted}{latex}
    % == custom style

    % adapt `long' glossarystyle so it takes the full width
    % inspired from see https://tex.stackexchange.com/a/118236 (old)
    % updated newglossarystyle skeleton https://tex.stackexchange.com/a/170627
    % \glsX commands http://tug.ctan.org/macros/latex/contrib/glossaries/glossaries-user.html
    \newglossarystyle{mylong}{%
    \setglossarystyle{longheader}% base this style on the `long' style
    \setlength\glsdescwidth{.7\textwidth} % set the description column width
    % no indent before the longtable:
    \setlength\LTleft{0pt}%
    % renew the table: both columns together now have \textwidth:
    \renewenvironment{theglossary}%
        {%
        \begin{longtable}
            {
            @{}>{\raggedright}
                p{\dimexpr\textwidth-\glsdescwidth-2\tabcolsep\relax}
            p{\glsdescwidth}
            @{}
            }
        }%
        {\end{longtable}}%
    % update the rows (especially the pagelist output)
    % glspostdescription adds the "." at the end of the description
    % ##2 is the pagelist
    \renewcommand*{\glossentry}[2]{%
    \glsentryitem{##1}\glstarget{##1}{\glossentryname{##1}} & % name
    \glossentrydesc{##1}\glspostdescription\space % description
        \emph{\footnotesize Page ##2}.\\ % pagelist
    }%
    }

    \setglossarystyle{mylong}
\end{minted}

\begin{listing}[ht!]
    \begin{minted}{latex}
        \newglossaryentry{derlin}{
            name={derlin},
            description={My developer name, which is simply my last name reversed: \emph{linder} $\rightarrow$ \emph{delrin}}
        }
        
        \newacronym{QPU}{QPU}{Quantum Processing Unit}

        \glsaddallunused % if you have some unused entries
    \end{minted}
    \caption{Example of \code{glossary.tex} content.}
    \label{lst:glossaries}
\end{listing}